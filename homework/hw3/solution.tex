\documentclass[12pt]{exam}

\usepackage[utf8]{inputenc}  % For UTF8 source encoding.
\usepackage{amsmath}  % For displaying math equations.
\usepackage{amsfonts} % For mathematical fonts (like \mathbb{E}!).
\usepackage{upgreek}  % For upright Greek letters, such as \upvarphi.
\usepackage{wasysym}  % For additional glyphs (like \smiley!).
\usepackage{mathrsfs} % For script text (hash families and universes).
\usepackage{enumitem}
\usepackage{graphicx}
% For document margins.
\usepackage[left=.8in, right=.8in, top=1in, bottom=1in]{geometry}
\usepackage{lastpage} % For a reference to the number of pages.
\usepackage[table,xcdraw]{xcolor}
\usepackage{pdfpages}
\usepackage{verbatim}

% TODO: Enter your name here :)
\newcommand*{\authorname}{Luis A. Perez}

\newcommand*{\duedate}{Wednesday, July 17th}
\newcommand*{\duetime}{11:59 pm}

% Fancy headers and footers
\headrule
\firstpageheader{EE 263\\Summer 2019}{Homework 3 \\ }{Due: \duedate\\at \duetime}
\runningheader{EE 263}{Homework 3}{\authorname}
\footer{}{\footnotesize{Page \thepage\ of \pageref{LastPage}}}{}

% Exam questions.
\newcommand{\Q}[1]{\question{\large{\textbf{#1}}}}
\qformat{}  % Remove formatting from exam questions.

% Useful macro commands.
\newcommand*{\bigtheta}[1]{\Theta\left( #1 \right)}
\newcommand*{\bigo}[1]{O \left( #1 \right)}
\newcommand*{\bigomega}[1]{\Omega \left( #1 \right)}
\newcommand*{\prob}[1]{\text{Pr} \left[ #1 \right]}
\newcommand*{\ex}[1]{\text{E} \left[ #1 \right]}
\newcommand*{\var}[1]{\text{Var} \left[ #1 \right]}

\newcommand*{\norm}[1]{\left\lVert #1 \right\rVert}
\newcommand*{\HH}{\mathscr{H}}   % Family of hash functions.
\newcommand*{\UU}{\mathscr{U}}   % Universe.
\newcommand*{\eps}{\varepsilon}  % Epsilon.


% Custom formatting for problem parts.
\renewcommand{\thepartno}{\roman{partno}}
\renewcommand{\partlabel}{\thepartno.}

% Framed answers.
\newcommand{\answerbox}[1]{
\begin{framed}
\hspace{\fill}
\vspace{#1}
\end{framed}}

\printanswers

\setlength\answerlinelength{2in} \setlength\answerskip{0.3in}

\begin{document}
\title{EE 263 Homework 3}
\author{\authorname}
\date{}
\maketitle
\thispagestyle{headandfoot}
\setcounter{MaxMatrixCols}{15}

\begin{questions}
%%%%%%%%%%%%%%%%%%%%%%%%%%%%%%%%%%%
\Q{Memory of a linear dynamical, time-invariant system}

  \begin{solution}
    \begin{enumerate}[label=(\alph*)]
      \item Let us first consider how we might check if a valid impose response of fixed size $M$ exists. The first thing to note is that the covolution operator given in the problem statent can actually be written as a linear system:
      \[
        \bar{y} = Ah
      \]
      where $\bar{y} \in \mathbb{R}^{T - M}, h \in \mathbb{R}^{M}$ and $A \in \mathbb{R}^{(T-M) \times M}$. Note that we remove the first $\{y_1, \cdots, y_{M}\}$. In fact, we can use the matrix $A$ as defined below:
      \begin{align*}
        A &=
          \begin{bmatrix}
          u_M & u_{M-1} & u_{M-2} & \cdots & u_1 \\ 
          u_{M+1} & u_{M} & u_{M-1} & \cdots & u_2 \\ 
          u_{M+2} & u_{M + 1} & u_{M} & \cdots & u_2 \\ 
          \vdots & \vdots & \vdots & \ddots \\
          u_{T-2} & u_{T-3} & u_{T-4} & \cdots & u_{T-M-1} \\
          u_{T-1} & u_{T-2} & u_{T-3} & \cdots & u_{T-M} \\
          \end{bmatrix} \\
        y_{-1} &= 
          \begin{bmatrix}
            y_{M+1} \\
            \vdots \\
            y_T
          \end{bmatrix} \\
        h &= 
          \begin{bmatrix}
            h_1 \\
            \vdots \\
            h_M
          \end{bmatrix}
      \end{align*}
      We can see by inspection above that $Ah$ performs the needed convolutions betweens $u$ and $h$ to obtain $\bar{y}$, by the properties of matrix multiplication (eg, to obtain $y_i$, we compute the dot product of the $i-M$-th row of $A$ with $h$, which is exactly what our convolution dictates, and for $i \leq M$, the convolution is not fully defined so we ignore them).

      In the problem statement, we're given the fact that $T > 2M$, so this means that there will always be at least $T-M > 2M -M = M$ rows in $A$, meaning it will always be a tall and skinny matrix. 

      As such, we can use the pseudo inverse to find the closests solution for $h$, computing:
      \[
        \bar{h} = (A^TA)^{-1}A^T\bar{y}
      \]
      Finally, once we've computed this $h$, we can re-compute that $\bar{y}$ given our inputs forming $A$, and see if this equals our what we started with. In other words, we have that:
      \[
        ||A\bar{h} - \bar{y}|| \leq \epsilon \implies M \text{ is a valid value} \tag{$\epsilon$ is needed to deal with floating point imprecision}
      \]
      The above gives us a way to check if $M$ is sufficient. To finallize our method, we simply iterate over the possible values of $M$ from $M = 1, \cdots \frac{T}{2} - 1$ in order until we find a valid value.

      \item Applying the process we described above, we find that $M = 7$ is the smallest value that works. We also have:
      \[
        h =
          \begin{bmatrix}
            0.63 \\
            0.27 \\
            0.02 \\
            0.37 \\
            0.96 \\
            0.95 \\
            0.46
          \end{bmatrix}
      \]
    \end{enumerate}
  \end{solution}
\end{questions}






















\end{document}
